\chapter{Технологическая часть}

\section{Выбор языка программирования}

Для реализации указанных алгоритмов был выбран язык C++, т.~к. данный язык предоставляет возможность работать с динамической памятью с помощью контейнерных классов.

Для визуализации данных использовался язык Python, т.~к. данный язык обладает большим выбором библиотек для выполнения этой задачи. В лабораторной работе использовалась библиотека matplotlib~\cite{plt}.

В качестве IDE был выбран Clion, т.~к. обладает функциями подсветки синтаксиса и автодополнения для обоих указанных языков. Также в данной IDE реализована поддержка средства автоматизированной сборки проектов cmake.

\section{Реализации алгоритмов}

В листингах ниже представлены реализации указанных алгоритмов:
\begin{itemize}[label=---]
	\item в листинге~\ref{lst:linear} --- реализация линейного поиска;
	\item в листинге~\ref{lst:binary} --- реализация бинарного поиска.
\end{itemize}

Данные реализации дополнительно возвращают число сравнений, необходимое для дальнейшего анализа.
\clearpage
\begin{lstlisting}[caption={Функция линейного поиска}, label=lst:linear]
	pair<int, int> linear_search(const vector<int> &a, int elem_to_find) {
		int i, cnt = 1;
		for (i = 0; i < a.size() && a[i] != elem_to_find; ++i, cnt++) {}
		return {(i == a.size() ? -1 : i), (i == a.size() ? cnt - 1 : cnt)};
	}
\end{lstlisting}
\begin{lstlisting}[caption={Функция бинарного поиска}, label=lst:binary]
	pair<int, int> binary_search(const vector<int> &a, int elem_to_find) {
		int l = -1, r = a.size(), m, cnt = 0;
		while (r - l > 1) {
			m = (r + l) / 2;
			if (a[m] == elem_to_find) {
				++cnt;
				return {m, cnt};
			} else if (a[m] > elem_to_find) {
				cnt += 2;
				r = m;
			} else {
				cnt += 2;
				l = m;
			}
		}
		return {-1, cnt};
	}
\end{lstlisting}

\section{Тестирование}
Для тестирования алгоритмов, тестовые случаи были разбиты на классы эквивалентности. В таблице~\ref{tab:lin_tests} приведены тесты для линейного поиска. В таблице~\ref{tab:bin_tests} для бинарного поиска соответственно.

\begin{table}[H]
	\centering
	\caption{Тестовые случаи для алгоритма линейного поиска}
	\begin{tabular}{|c|c|c|c|c|c|}
		\hline
		\multirow[c]{2}{*}{\shortstack{Классы\\эквивалентности}} & \multirow[c]{2}{*}{Входные данные} & \multicolumn{2}{c|}{\shortstack{Ожидаемые\\выходные данные}} & \multicolumn{2}{c|}{\shortstack{Полученные\\выходные данные}} \\
		\hhline{|~|~|-|-|-|-|}
		&& Индекс & \shortstack{Число\\сравнений} & Индекс & \shortstack{Число\\сравнений}\\
		\hline
		\shortstack{Искомый элемент\\первый} & \shortstack{1, 2, 3, 4, 5\\1} & 0 & 1 & 0 & 1 \\
		\hline 
		\shortstack{Искомый элемент\\последний} & \shortstack{1, 2, 3, 4, 5\\5} & 4 & 5 & 4 & 5 \\
		\hline 
		\shortstack{Искомый элемент\\центральный} & \shortstack{1, 2, 3, 4, 5\\3} & 2 & 3 & 2 & 3 \\
		\hline 
		\shortstack{Искомый элемент\\отсутствует} & \shortstack{1, 2, 3, 4, 5\\-1} & -1 & 5 & -1 & 5 \\
		\hline
		\shortstack{Массив пуст} & \shortstack{<<>>\\5} & -1 & 0 & -1 & 0 \\
		\hline  
		\end{tabular}
	\label{tab:lin_tests}
\end{table}
\begin{table}[H]
	\centering
	\caption{Тестовые случаи для алгоритма бинарного поиска}
	\begin{tabular}{|c|c|c|c|c|c|}
		\hline
		\multirow[c]{2}{*}{\shortstack{Классы\\эквивалентности}} & \multirow[c]{2}{*}{Входные данные} & \multicolumn{2}{c|}{\shortstack{Ожидаемые\\выходные данные}} & \multicolumn{2}{c|}{\shortstack{Полученные\\выходные данные}} \\
		\hhline{|~|~|-|-|-|-|}
		&& Индекс & \shortstack{Число\\сравнений} & Индекс & \shortstack{Число\\сравнений}\\
		\hline
		\shortstack{Искомый элемент\\первый} & \shortstack{1, 2, 3, 4, 5\\1} & 0 & 3 & 0 & 3 \\
		\hline 
		\shortstack{Искомый элемент\\последний} & \shortstack{1, 2, 3, 4, 5\\5} & 4 & 5 & 4 & 5 \\
		\hline 
		\shortstack{Искомый элемент\\центральный} & \shortstack{1, 2, 3, 4, 5\\3} & 2 & 1 & 2 & 1 \\
		\hline 
		\shortstack{Искомый элемент\\отсутствует} & \shortstack{1, 2, 3, 4, 5\\-1} & -1 & 4 & -1 & 4 \\
		\hline
		\shortstack{Массив пуст} & \shortstack{<<>>\\5} & -1 & 0 & -1 & 0 \\
		\hline  
	\end{tabular}
	\label{tab:bin_tests}
\end{table}

Для проведения тестирования использовалась библиотека Google Tests~\cite{gtest}. 
\section*{Вывод}

Были реализованы алгоритмы поиска элемента в массиве и проведено их тестирование. Все тесты были успешно пройдены.

\clearpage