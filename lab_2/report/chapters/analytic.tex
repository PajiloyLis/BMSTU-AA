\chapter{Аналитическая часть}

Ниже представлены два основных и наиболее распространенных~\cite{aa_lec} метода поиска элемента в массиве:
\begin{itemize}[label=---]
	\item линейный поиск;
	\item бинарный поиск.
\end{itemize}

\section{Линейный поиск}

\subsection{Описание алгоритма}

Линейный поиск --- метод поиска, при котором последовательно просматриваются и сравниваются с искомым все элементы массива.

Соответственно для линейного поиска возможен $n+1$ исход, где $n$ --- количество элементов в массиве. При этом лучший случай наступает, когда элемент обнаруживается на первой же позиции, т.~e. производится единственное сравнение. Худших случаев два: элемент найден на последней позиции или не найден вообще. При этом оба случая обойдутся в $n$ сравнений.

Пусть $k_1$ --- число операций производимых линейным поиском до прохода по всем элементам, например, объявление и определение начального значения индекса. Пусть также $k_2$ --- число операций производимых на каждой итерации просмотра очередного элемента, например, операции сравнения и инкремента индекса. Тогда для линейного поиска:
\begin{equation}
	\label{eq:lin_best_case}
	f_{best}(n) = k_1 + k_2
\end{equation}
\begin{equation}
	\label{eq:lin_worst_case}
	f_{worst}(n) = k_1 + n \cdot k_2
\end{equation}
\begin{equation}
	\label{eq:lin_standart_case}
	f_{mean}(n) = k_1 + \displaystyle\sum_{i\in\Omega}^{} p_i \cdot \delta_i = k_1 + k_2 \cdot (1 + \frac{n}{2} - \frac{1}{n + 1})
\end{equation}

где $\Omega$ --- множество всех возможных случаев, $p_i$ --- вероятность $i$- случая, для $\forall i\ p_i = \frac{1}{n}$, т.~к. все случаи равновозможны, $\delta_i$ --- количество операций, которые необходимо выполнить для нахождения элемента в $i$-ом случае. 

Формулы~\ref{eq:lin_best_case},~\ref{eq:lin_worst_case} и~\ref{eq:lin_standart_case} отражают лучший, худший и средний по числу операций случаи соответственно. Из худшего случая следует, что асимптотика такого поиска $O(n)$.

\section{Бинарный поиск}
\subsection{Описание алгоритма}

Алгоритм бинарного поиска применим только к упорядоченным массивам, в связи с чем, возникает необходимость предварительной сортировки, если таковая возможна.
Основная идея алгоритма --- возможность отбросить половину отрезка поиска, после сравнения его середины с искомым элементом~\cite{Sedgewick03}. Так, если массив упорядочен по неубыванию, и средний элемент отрезка элемента больше искомого, поиск необходимо продолжать в левой половине с меньшими среднего элементами. Иначе в правой с большими соответственно.

В этом случае последовательное уменьшение отрезка поиска можно представить в виде бинарного дерева, высота которого будет не более $\log_2{n}$. Таким образом, бинарный поиск не проверяет более 
\begin{equation}
	\label{eq:bin_cnt_elems}
	\lfloor \log_2{n} \rfloor + 1
\end{equation}
элементов~\cite{Sedgewick03}. Откуда асимптотика $O(\log{n})$.

\section*{Вывод}

Рассмотрены алгоритмы линейного и бинарного поисков элемента в массиве, приведены их асимптотики и оценки количества операций сравнения.

\clearpage