\ssr{ЗАКЛЮЧЕНИЕ}

Целью данной работы была организация параллельных вычислений по конвейерному принципу с использованием нативных потоков.

В ходе лабораторной работы были выполнены следующие задачи:
\begin{enumerate}[label={\arabic*)}]
	\item разработано ПО осуществляющее обработку html страниц полученных от ПО, разработанного в рамках ЛР №4, и сохраняющее обработанные данные;
	\item проведено тестирование разработанного ПО;
	\item проведены замеры среднего времени существования задач, ожидания в каждой из очередей и обработки на каждой из стадий.
\end{enumerate}

По итогам анализа было выявлено, что поиск данных в html файле занимает наибольшее время среди выполнения этапов конвейера. Соответственно, ожидание в очереди на обработку содержимого файла занимает также максимально время среди времен ожидания.