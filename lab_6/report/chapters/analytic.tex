\chapter{Аналитическая часть}

Задача коммивояжера~---~задача оптимизации, заключающаяся в поиске в графе пути (или цикла) минимальной стоимости, проходящего через каждую вершину только один раз. Таким образом задача коммивояжера сводится к задаче поиска гамильтонова пути (или цикла) минимальной стоимости. Задача коммивояжера является np-трудной задачей, соответственно, на данный момент не доказана возможность ее решения за полиномиальное время~\cite{Mudrov69}.

В работе рассматриваются два возможных метода решения задачи коммивояжера: полный перебор и метод на основе муравьиного алгоритма. 
\section{Метод на основе полного перебора}

\subsection{Описание алгоритма}

Данный алгоритм решает задачу методом <<грубой силы>>: перебор всех возможных перестановок вершин и поиск минимального веса пути, образованного вершинами в перестановке. Данный алгоритм для любого графа получит правильный ответ, однако его асимптотика позволяет использовать его лишь для графов с небольшим числом вершин~\cite{Levitin06}.

\subsection{Асимптотика}

Число перестановок из $n$ различных элементов составляет $n!$, соответственно, асимптотика алгоритма составляет $O(n!)$.

\section{Метод на основе муравьиного алгоритма}

\subsection{Описание алгоритма}

Алгоритм основывается на поведении муравьев при поиске пути от колонии к источнику пищи. 

Изначально в каждой вершине графа располагается по одному муравью. Вводится матрица феромонов: $T = (\tau_{i, j})_{i, j = \overline{1, n}}$, где $n$ --- число вершин в графе. Каждое $\tau_{i, j}$ обозначает количество феромона на ребре из вершины $i$ в вершину $j$. Для исключения возможности обнуления вероятности перехода в еще не посещенную вершину вводится некоторое минимальное количество феромона $q_{min}$. Изначально также полагается $\tau_{i, j} = q_{min}, \forall i, j  = \overline{1, n}$.

В каждый из $d$ дней каждый муравей пытается построить маршрут. Если муравей оказывается в тупике, т.~е. не может из вершины, в которой он сейчас находится перейти в еще не посещенную вершину, или найденный им маршрут имеет большую стоимость, чем найденный им ранее, такой маршрут не учитывается в обновлении феромонов. Иначе в соответствии с найденными маршрутами обновляются феромоны. В системе с элитными муравьями, в конце каждого дня вносится дополнительная доза феромона на маршруты минимальной длины. По истечении $d$ дней выбирается наилучший маршрут.

Вероятность или желание перехода муравья $k$ из текущей вершины $i$ в смежную с ней вершину $j$ в день $t$ определяется по формуле~\ref{eq:probability}.
\begin{equation}
	\label{eq:probability}
    P_{i, j}(t, k) = \begin{cases}
        \frac{\eta_{i,j}^{\alpha} \cdot \tau_{i, j}(t)^{\beta}}{\displaystyle\sum_{j\notin Y_{k}(t)}\eta_{i, j}^{\alpha}\cdot\tau_{i,j}(t)^{\beta}}, & j \notin Y_{k}(t) \\
        0&, j \in Y_{k}(t)
    \end{cases}
\end{equation}
где $\eta_{i,j}=\frac{1}{d_{i,j}}$~---~величина обратно пропорциональная весу ребра, $\tau_{i,j}(t)$~---~количество феромона на ребре из вершины $i$ в вершину $j$ в день $t$, $\alpha \in [0, 1]$~---~величина, называемая коэффициентом <<жадности>>, $\beta = 1 - \alpha$~---~величина, называемая коэффициентом <<стадности>>,  $Y_{k}(t)$ - множество ребер входящих в маршрут муравья на данный момент.

Таким образом выбор следующей вершины осуществляется среди всех еще не посещенных, смежных с данной. Выбор осуществляется случайным образом, однако, <<случайность>> распределена в соответствии, с ранее посчитанными вероятностями.

В конце дня необходимо обновить феромон на каждом ребре. Обновление осуществляется в соответствии с формулой~\ref{eq:global_update}.
\begin{equation}
	\label{eq:global_update}
	\tau_{i, j} = (1 - \rho) \cdot \tau_{i, j}(t) + \Delta \tau_{i, j}(t)
\end{equation}
где $\rho$ - коэффициент испарения, $\tau_{i, j}(t)$~---~количество феромона на ребре из вершины $i$ в вершину $j$ в день $t$, $\Delta \tau_{i, j}(t)$~---~ изменение феромона на ребре из вершины $i$ в вершину $j$ в день $t$ в связи с полученными муравьями путями.
$\Delta \tau_{i, j}(t)$ определяется в соответствии с формулами~\ref{eq:local_update},~\ref{eq:local_update_1}.
\begin{equation}
	\label{eq:local_update}
	\Delta \tau_{i, j}(t)= \displaystyle\sum_{k=1}^{n} \Delta \tau_{i, j}(t, k)
\end{equation}
\begin{equation}
	\label{eq:local_update_1}
	\Delta \tau_{i, j}(t, k) = \begin{cases}
		0, & \text{если муравей }k\text{ на итерации }t\text{ не проходил ребро из вершины }i\text{ в }j\\
		\frac{Q}{L_{k}(t)}, & \text{иначе}
	\end{cases}
\end{equation} 
где $Q$~---~дневной запас феромона, $L_{k}(t)$~---~длина пути построенного муравьем $k$ в день $t$.

Все обновления производятся, если найденный в текущий день путь короче, чем все пути найденные до этого. Иначе обновления производятся в соответствии с предыдущим кратчайшим путем.

Метод на основе муравьиного алгоритма имеет меньшую асимптотическую сложность, чем полный перебор за $O(n!)$, однако данный метод не всегда находит точный ответ~\cite{Dorigo04}.

\section*{Вывод}

В данной части были рассмотрены два метода решения задачи коммивояжера: метод на основе полного перебора и метод на основе муравьиного алгоритма.

\clearpage