\ssr{ЗАКЛЮЧЕНИЕ}

Целью данной работы являлся анализ методов решения задачи коммивояжера: полный перебор и метод на основе муравьиного алгоритма.
В ходе лабораторной работы были выполнены следующие задачи:
\begin{enumerate}[label={\arabic*)}]
	\item сформулирована постановка задачи коммивояжера;
	\item описаны методы решения задачи коммивояжера;
	\item реализованы указанные методы;
	\item проведена параметризация метода на основе муравьиного алгоритма и выявлены оптимальные параметры;
	\item проведены замеры и анализ временной эффективности указанных методов.
\end{enumerate}

По итогам анализа было выявлено, что оптимальными для муравьиного алгоритма являются следующие параметры:
\begin{itemize}
	\item $\alpha \in [0,\ 0.4]$;
	\item $\beta \in [0.6,\ 0.1]$;
	\item $\rho \in [0.1,\ 0.2]$;
	\item $\text{число дней} > 100$;
	\item $\text{число элитных муравьев} \in [5, 15]$.
\end{itemize}

В результате анализа временных затрат было определено, что метод на основе муравьиного алгоритма превосходит полный перебор в скорости выполнения для графов с числом вершин не менее 9.

Поставленная цель исследования методов решения задачи коммивояжера была достигнута.
\clearpage
