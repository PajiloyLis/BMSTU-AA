\chapter{Технологическая часть}

\section{Выбор языка программирования}

Для реализации указанных алгоритмов был выбран язык C++, т.~к. данный язык предоставляет возможность работать с динамической памятью с помощью контейнерных классов.

Для визуализации данных использовался язык Python, т.~к. данный язык обладает широким выбором библиотек для выполнения этой задачи. В лабораторной работе использовалась библиотека matplotlib~\cite{plt}.

В качестве IDE был выбран Clion, т.~к. обладает функциями подсветки синтаксиса и автодополнения для обоих указанных языков. Также в данной IDE реализована поддержка средства автоматизированной сборки проектов cmake.

\section{Реализации алгоритмов}

В листингах ниже представлены реализации указанных алгоритмов:
\begin{itemize}[label=---]
	\item в листинге~\ref{lst:std_imp}~---~реализация алгоритма полного перебора;
	\item в листинге~\ref{lst:win_imp}~---~функция вычисления дневной дозы феромона для данного графа;
	\item в листинге~\ref{lst:opt_win_imp}~---~реализация муравьиного алгоритма;
	\item в листинге~\ref{lst:connectivity}~---~функция проверки графа на связность;
	\item в листинге~\ref{lst:next_permutation}~---~функция генерации следующей в лексикографическом порядке перестановки.
\end{itemize}

\clearpage
\begin{lstlisting}[caption={Алгоритм полного перебора}, label=lst:std_imp]
	pair<double, vector<int>> complete_bust(const vector<vector<double>> &g) {
		int n = g.size();
		vector<int> permutation(n);
		
		vector<int> path;
		double sum = 0, cur_ans = -1;
		
		for (int i = 0; i < n; ++i)
		permutation[i] = i;
		
		do {
			for (int i = 1; i < permutation.size(); ++i) {
				sum += g[permutation[i - 1]][permutation[i]];
			}
			if (cur_ans == -1 || cur_ans > sum) {
				path = permutation;
				cur_ans = sum;
			}
			sum = 0;
		} while (next_permutation_(permutation.begin(), permutation.end()));
		return {cur_ans, path};
	}
\end{lstlisting}
\begin{lstlisting}[caption={Вычисление дневной дозы феромона}, label=lst:win_imp]
	double calc_q(const vector<vector<double>> &g) {
		double q = 0;
		int count = 0;
		for (const auto &i: g)
		for (double j: i)
		if (j < INF) {
			q += j;
			++count;
		}
		return q / count;
	}
\end{lstlisting}
\clearpage
\begin{lstlisting}[caption={Муравьиный алгоритм}, label=lst:opt_win_imp]
	pair<double, vector<int>>
	ant_algorithm(const vector<vector<double>> &g, int days_cnt, double a, double b, double r, int elite_cnt) {
		int n = g.size();
		double q = calc_q(g), min_q = 1;
		vector<int> ans(n);
		vector<vector<int>> paths(n, vector<int>(0));
		vector<double> paths_lengths(n, INF);
		vector<vector<double>> pheromones(n, vector<double>(n, min_q));
		vector<bool> visited;
		int visited_cnt, cur_pos;
		vector<double> wishes(n);
		vector<int> cur_path;
		double total_wish, probability, best_path_length, ans_path_length = INF, cur_length;
		for (int t = 0; t < days_cnt; ++t) {
			for (int ant = 0; ant < n; ++ant) {
				cur_path.clear();
				cur_length = 0;
				visited.clear();
				visited.resize(n, false);
				cur_pos = ant;
				cur_path.push_back(ant);
				visited[cur_pos] = true;
				visited_cnt = 1;
				while (visited_cnt < n) {
					total_wish = 0;
					for (int i = 0; i < n; ++i)
					total_wish += ((visited[i] || g[cur_pos][i] >= INF - EPS) ? 0 : pow(1 / g[cur_pos][i], a) * pow(pheromones[cur_pos][i], b));
					if (abs(total_wish) < EPS) {
						cur_length=INF;
						break;
					}
					for (int i = 0; i < n; ++i)
					wishes[i] = ((visited[i] || g[cur_pos][i] >= INF - EPS) ? 0 : pow(1 / g[cur_pos][i], a) * pow(pheromones[cur_pos][i], b) / total_wish);
					probability = (double) rnd() / rnd.max();
					for (int i = 0; i < n; ++i) {
						if (probability >= wishes[i])
						probability -= wishes[i];
						else {
							cur_length += g[cur_pos][i];
							cur_pos = i;
							cur_path.push_back(cur_pos);
							visited_cnt++;
							visited[cur_pos] = true;
						}
					}
				}
				if (cur_length < paths_lengths[ant]) {
					paths_lengths[ant] = cur_length;
					paths[ant] = cur_path;
				}
			}
			for (int i = 0; i < n; ++i) {
				for (int j = 0; j < n; ++j) {
					pheromones[i][j] *= (1 - r);
					if (pheromones[i][j] < min_q)
						pheromones[i][j] = min_q;
				}
			}
			for (int i = 0; i < n; ++i)
			for (int j = 1; j < paths[i].size(); ++j) {
				pheromones[paths[i][j - 1]][paths[i][j]] += q / paths_lengths[i];
				pheromones[paths[i][j]][paths[i][j - 1]] += q / paths_lengths[i];
			}
			best_path_length = paths_lengths[0];
			for (int i = 0; i < n; ++i) {
				best_path_length = min(best_path_length, paths_lengths[i]);
			}
			for (int i = 0; i < n; ++i) {
				if (abs(paths_lengths[i] - best_path_length) < EPS) {
					for (int j = 1; j < paths[i].size(); ++j) {
						pheromones[paths[i][j - 1]][paths[i][j]] += elite_cnt * q / paths_lengths[i];
						pheromones[paths[i][j]][paths[i][j - 1]] += elite_cnt * q / paths_lengths[i];
					}
				}
			}
		}
		ans_path_length = paths_lengths[0];
		ans = paths[0];
		for (int i = 0; i < n; ++i) {
			paths_lengths[i]);
			if (paths_lengths[i] < ans_path_length) {
				ans_path_length = paths_lengths[i];
				ans = paths[i];
			}
		}
		return {ans_path_length, ans};
	}
\end{lstlisting}
\begin{lstlisting}[caption={Функция проверки графа на связность и dfs}, label={lst:connectivity}]
	bool check_connectivity(const vector<vector<double>> &g) {
		vector<bool> used(g.size());
		dfs(g, used, 0);
		for (int i = 0; i < g.size(); ++i)
		if (!used[i])
		return false;
		return true;
	}
	
	void dfs(const vector<vector<double>> &g, vector<bool> &used, int v) {
		used[v] = true;
		for (int u = 0; u < g.size(); ++u)
		if (!used[u] && g[v][u] < INF)
		dfs(g, used, u);
	}
\end{lstlisting}
\clearpage
\begin{lstlisting}[caption={Функция генерации следующей в лексикографическом порядке перестановки}, label=lst:next_permutation]
bool next_permutation_(vector<int>& nums) {
	int n = nums.size();
	if (n <= 1) return false;
	
	int i = n - 2;
	while (i >= 0 && nums[i] >= nums[i + 1]) {
		--i;
	}
	
	if (i < 0) {
		std::reverse(nums.begin(), nums.end());
		return false;
	}
	
	int j = n - 1;
	while (nums[j] <= nums[i]) {
		--j;
	}
	
	std::swap(nums[i], nums[j]);
	
	std::reverse(nums.begin() + i + 1, nums.end());
	
	return true;
}
\end{lstlisting}

\section{Тестирование}
Тестовые случаи для тестирования функции проверки графа на связность представлены в таблице~\ref{tab:connectivity_test}.

\begin{table}[H]
	\centering
	\caption{Тестовые случаи для функции проверки связности}
	\begin{tabular}{|c|c|c|c|}
		\hline
		\shortstack{Класс\\эквивалентности} &
		\shortstack{Входные\\данные} & \shortstack{Ожидаемые\\\shortstack{выходные\\данные}} & \shortstack{Полученные\\\shortstack{выходные\\данные}}\\
		\hline
		Несвязный граф &
		$
		\begin{pmatrix}
			0 & 10 & INF & INF & INF \\
			10 & 0 &  INF & INF & INF\\
			INF & INF & 0 & 15 & 12\\
			INF & INF & 15 & 0 & 13\\
			INF & INF & 12 & 13 & 0\\
		\end{pmatrix}
		$
		&
		false
		&
		false
		\\
		\hline 
		Связный граф &
		$
		\begin{pmatrix}
			0 & 12 & 13\\
			12 & 0 & 15\\
			13 & 15 & 0\\
		\end{pmatrix}
		$
		&
		true
		&
		true
		\\
		\hline 
	\end{tabular}
	\label{tab:connectivity_test}
\end{table}

Тестовые данные для функции вычисления начального количества феромона представлены в таблице~\ref{tab:calc_q_test}.

\begin{table}[H]
	\centering
	\caption{Тестовые случаи для вычисления начального количества феромона}
	\begin{tabular}{|c|c|c|c|}
		\hline
		\shortstack{Класс\\эквивалентности} &
		\shortstack{Входные\\данные} & \shortstack{Ожидаемые\\\shortstack{выходные\\данные}} & \shortstack{Полученные\\\shortstack{выходные\\данные}}\\
		\hline
		Несвязный граф &
		$
		\begin{pmatrix}
			0 & 25 & INF & INF & INF \\
			25 & 0 &  INF & INF & INF\\
			INF & INF & 0 & 15 & 12\\
			INF & INF & 15 & 0 & 13\\
			INF & INF & 12 & 13 & 0\\
		\end{pmatrix}
		$
		&
		10
		&
		10
		\\
		\hline 
		Связный граф &
		$
		\begin{pmatrix}
			0 & 12 & 13\\
			12 & 0 & 20\\
			13 & 20 & 0\\
		\end{pmatrix}
		$
		&
		10
		&
		10
		\\
		\hline 
	\end{tabular}
	\label{tab:calc_q_test}
\end{table}

Тестовые данные для проверки метода полного перебора представлены в таблице~\ref{tab:complete_burst_test}.

\begin{table}[H]
	\centering
	\caption{Тестовые случаи для метода полного перебора}
	\begin{tabular}{|c|c|c|c|}
		\hline
		\shortstack{Класс\\эквивалентности} &
		\shortstack{Входные\\данные} & \shortstack{Ожидаемые\\\shortstack{выходные\\данные}} & \shortstack{Полученные\\\shortstack{выходные\\данные}}\\
		\hline
		Полный граф &
		$
		\begin{pmatrix}
			0 & 8 & 10 & 9\\
			8 & 0 & 5 & 3\\
			10 & 5 & 0 & 2\\
			9 & 3 & 2 & 0\\
		\end{pmatrix}
		$
		&
		\shortstack{Длина: 13\\\shortstack{Путь:\\1, 2, 4, 3}}
		&
		\shortstack{Длина: 13\\\shortstack{Путь:\\1, 2, 4, 3}}
		\\
		\hline 
		\shortstack{Связный не\\полный граф} &
		$
		\begin{pmatrix}
			0 & 6 & INF & 5 & INF\\
			6 & 0 & 7 & 8 & INF\\
			INF & 7 & 0 & INF & 1\\
			5 & 8 & INF & 0 & 13\\
			INF & INF & 1 & 13 & 0\\
		\end{pmatrix}
		$
		&
		\shortstack{Длина: 19\\\shortstack{Путь:\\4, 1, 2, 3, 5}}
		&
		\shortstack{Длина: 19\\\shortstack{Путь:\\4, 1, 2, 3, 5}}
		\\
		\hline 
	\end{tabular}
	\label{tab:complete_burst_test}
\end{table}

Тестовые данные для метода на основе муравьиного алгоритма представлены в таблице~\ref{tab:ant_test}.

\begin{table}[H]
	\centering
	\caption{Тестовые случаи для муравьиного алгоритма}
	\begin{tabular}{|c|c|c|c|}
		\hline
		\shortstack{Класс\\эквивалентности} &
		\shortstack{Входные\\данные} & \shortstack{Ожидаемые\\\shortstack{выходные\\данные}} & \shortstack{Полученные\\\shortstack{выходные\\данные}}\\
		\hline
		Полный граф &
		$
		\begin{pmatrix}
			0 & 8 & 10 & 9\\
			8 & 0 & 5 & 3\\
			10 & 5 & 0 & 2\\
			9 & 3 & 2 & 0\\
		\end{pmatrix}
		$
		&
		\shortstack{Длина: 13\\\shortstack{Путь:\\1, 2, 4, 3}}
		&
		\shortstack{Длина: 13\\\shortstack{Путь:\\1, 2, 4, 3}}
		\\
		\hline 
		\shortstack{Связный не\\полный граф} &
		$
		\begin{pmatrix}
			0 & 6 & INF & 5 & INF\\
			6 & 0 & 7 & 8 & INF\\
			INF & 7 & 0 & INF & 1\\
			5 & 8 & INF & 0 & 13\\
			INF & INF & 1 & 13 & 0\\
		\end{pmatrix}
		$
		&
		\shortstack{Длина: 19\\\shortstack{Путь:\\4, 1, 2, 3, 5}}
		&
		\shortstack{Длина: 19\\\shortstack{Путь:\\4, 1, 2, 3, 5}}
		\\
		\hline 
	\end{tabular}
	\label{tab:ant_test}
\end{table}

Для проведения тестирования использовалась библиотека Google Tests~\cite{gtest}. 
\section*{Вывод}

Были реализованы методы решения задачи коммивояжера. Проведено тестирование метода на основе полного перебора, а также вспомогательных функций. Все тесты были успешно пройдены.

\clearpage