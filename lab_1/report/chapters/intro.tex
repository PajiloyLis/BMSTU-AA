\ssr{ВВЕДЕНИЕ}

Расстояние Левенштейна --- метрика, показывающая, разницу между двумя последовательностями символов. Данное редакционное расстояние основано на минимальном количестве операций вставки, удаления, изменения символов необходимом для преобразования одной строки в другую.

Впервые задача была поставлена в 1965 году Владимиром Левенштейном при исследовании 0~--~1 последовательностей~\cite{levenstein65}. Позднее была обобщена на произвольные алфавиты. В дальнейшем Фредерик Дамерау добавил операцию транспозиции, показав, что 80\% ошибок при печати представлены этими операциями~\cite{damerau64}.

Нахождение редакционного расстояния применяется для решения следующих задач:
\begin{itemize}[label=---]
	\item исправление ошибок при вводе текста в поисковых системах, базах данных, при автоматическом распознавании речи;
	\item сравнение текстовых файлов утилитой diff;
	\item сравнение хромосом, геном, белков в биоинформатике.
\end{itemize}

Цель данной работы --- исследование алгоритмов поиска расстояний Левенштейна и Дамерау---Левенштейна. 

Задачи лабораторной работы:
\begin{enumerate}[label={\arabic*)}]
	\item описать алгоритмы поиска расстояний Левенштейна и Дамерау---Левенштейна;
	\item реализовать следующие алгоритмы:
	\begin{itemize}[label=---]
		\item нерекурсивный алгоритм поиска расстояния Левенштейна;
		\item рекурсивный алгоритм поиска расстояния Левенштейна;
		\item рекурсивный алгоритм поиска расстояния Левенштейна с мемоизацией;
		\item нерекурсивный алгоритм поиска расстояния Дамерау---Левенштейна;
	\end{itemize}
	\item определить подходящие для измерения процессорного времени инструменты;
	\item провести анализ временных и емкостных затрат различных реализаций.
\end{enumerate}
\clearpage