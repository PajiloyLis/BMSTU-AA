\chapter{Аналитическая часть}

\section{Расстояние Левенштейна}

\subsection{Описание алгоритма}

Расстояние Левенштейна --- одна из разновидностей редакционного расстояния~\cite{navaro01}, метрика, которая отражает разницу между двумя строками. Расстояние Левенштейна показывает наименьшее количество операций вставки, удаления, замены символа необходимых для преобразования одной строки к другой.

В общем случае каждая операция имеет свою собственную стоимость:
\begin{itemize}[label=---]
	\item $\omega(a, b),\ a \neq b$ --- стоимость замены буквы $a$ на $b$;
	\item $\omega(\lambda, a)$ --- стоимость вставки буквы $a$;
	\item $\omega(a, \lambda)$ --- стоимость удаления буквы $a$.
\end{itemize}

В дальнейшем стоимость каждой из операций принимается равной 1:
\begin{itemize}[label=---]
	\item $\omega(a, b) = 1,\ a \neq b$;
	\item $\omega(\lambda, a) = 1$;
	\item $\omega(a, \lambda) = 1$.
\end{itemize}

Отдельно введен случай совпадения символов. Его стоимость равна 0 --- $\omega(a, a) = 0$.

Пусть функция $D(i, j)$ - расстояние Левенштейна между префиксами рассматриваемых строк, т.~e. $s_1[1\dots i]$ и $s_2[1\ldots j]$. Тогда расстояние Левенштейна для двух строк $s_1$ и $s_2$ длины $n$ и $m$ соответственно, может быть вычислено по рекурентной формуле:
\begin{equation}
	\label{eq:levenstein}
	D(i, j) = 
	\begin{cases}
		0 & i=0,~j=0\\
		i &i > 0,~j = 0\\
		j &i = 0,~j > 0\\
		min\begin{cases}
			D(i-1, j)\\
			D(i, j-1)\\
			D(i-1, j-1) + match(s_1[i], s_2[j])
		\end{cases} & i > 0,~j > 0
	\end{cases}
\end{equation}

где $match(a, b)$ - функция совпадения символов:
\begin{equation}
	\label{eq:levenstein_match}
	match(a, b) = 
	\begin{cases}
		0 &a \neq b\\
		1 &a = b
	\end{cases}
\end{equation}
\subsection{Рекурсивная реализация}

Данная реализация представляет собой в точности имплементацию формулы~(\ref{eq:levenstein}).

\subsection{Рекурсивная реализация с мемоизацией}

Для вычисления минимума в формуле~(\ref{eq:levenstein}) происходит многократный пересчет промежуточных значений. Чтобы избежать повторных вычислений, введена матрица $A[(n + 1) \times (m + 1)]$ в ячейке $A_{i, j}$ которой будет записываться значение $D(i, j)$ при первом расчете. При повторной необходимости вычислить $D(i, j)$, значение будет подставлено из матрицы.
\subsection{Нерекурсивная реализация}

Рекурсивная реализация с мемоизацией имеет дополнительные временные и емкостные затраты на рекурсивные вызовы. Поэтому для нахождения расстояния Левенштейна был предложен подход динамического программирования. При таком подходе предполагается итерационное заполнение матрицы $A$. При этом порядок обхода возможен как по строкам, так и по столбцам, так как все необходимые значения для текущего шага итерации уже будут вычислены. Расстояние Левенштейна для исходных строк будет при этом находиться в $A_{n, m}$.

\section{Расстояние Дамерау---Левенштейна}
\subsection{Описание алгоритма}

Расстояние Дамерау --- Левенштейна --- также одна из разновидностей редакционных расстояний. Представляет собой модификацию расстояния Левенштейна, путем добавления операции транспозиции, т.~е. перестановки букв в строке для совпадения с другой строкой.

Расстояние Дамерау---Левенштейна также может быть вычислено по рекурентной формуле:

\begin{equation}
	\label{eq:damerau_levenstein}
	DL(i, j) = 
	\begin{cases}
		0 & i=0,~j=0\\
		i &i > 0,~j = 0\\
		j &i = 0,~j > 0\\
		min\begin{cases}
			DL(i-1, j)\\
			DL(i, j-1)\\
			DL(i-1, j-1) + match(s_1[i], s_2[j])\\
			DL(i-2, j-2) + 1
		\end{cases} & \begin{aligned}
			&i > 1,~j > 1,\\
			&s_1[i] = s_2[j - 1],\\
			&s_1[i - 1] = s_2[j]
		\end{aligned}\\
		min\begin{cases}
			DL(i-1, j)\\
			DL(i, j-1)\\
			DL(i-1, j-1) + match(s_1[i], s_2[j])
		\end{cases} & \text{иначе}
	\end{cases}
\end{equation}

где $match(a, b)$ --- по-прежнему функция совпадения символов~(\ref{eq:levenstein_match}), $DL(i, j)$ --- расстояние Дамерау --- Левенштейна для префиксов $s_1[1\ldots i]$ и $s_2[1\ldots j]$.

\subsection{Нерекурсивная реализация}

Данный алгоритм аналогичен нерекурсивному поиску расстояния Левенштейна. В ячейку $A_{i, j}$ матрицы $A[(n + 1) \times (m + 1)]$ записывается значение $DL(i, j)$. Заполнение также может производиться по строкам или столбцам, а окончательный ответ по-прежнему будет в $A_{n, m}$.

\section*{Вывод}

Рассмотрены алгоритмы нахождения расстояний Левенштейна и Дамерау---Левенштейна. Так как оба алгоритма заданы рекурентными формулами, возможна их рекурсивная или итеративная реализация с использованием подхода динамического программирования.

\clearpage