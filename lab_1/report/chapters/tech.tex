\chapter{Технологическая часть}

\section{Выбор языка программирования}

Для реализации указанных алгоритмов и выполнения лабораторной работы был выбран язык C++, т.~к. программа на нем может быть запущена на микроконтроллере для замеров времени работы алгоритма. Также данный язык предоставляет возможность работать с динамической памятью с помощью контейнерных классов.

Была выбрана IDE Arduino-IDE, т.~к. она дает возможность запускать код на выбранном языке напрямую на микроконтроллере, обладает функциями автодополнения и подсветки синтаксиса.

\section{Реализации алгоритмов}

В листингах ниже представлены реализации указанных алгоритмов:
\begin{itemize}[label=---]
	\item в листинге~\ref{lst:lev_rec} --- рекурсивная реализация алгоритма поиска расстояния Левенштейна;
	\item в листинге~\ref{lst:lev_rec_mem_1} и~\ref{lst:lev_rec_mem_2} --- рекурсивные реализации алгоритма с мемоизацией. Первая функция является интерфейсной, вторая необходима для использования матрицы с уже рассчитанными значениями;
	\item в листинге~\ref{lst:lev_iter} --- итерационная реализация с использованием подхода динамического программирования;
	\item в листинге~\ref{lst:dam_lev_iter} --- итерационная реализация алгоритма поиска расстояния Дамерау---Левенштейна с использованием подхода динамического программирования.
\end{itemize}
\clearpage
\begin{lstlisting}[caption={Рекурсивная функция поиска расстояния Левенштейна}, label=lst:lev_rec]
	int levenstein_recursion(const string &s1, const string &s2) {
		if (s1.empty())
			return static_cast<int>(s2.size());
		else if (s2.empty())
			return static_cast<int>(s1.size());
		else {
			int ans = min(levenstein_recursion(s1.substr(0, s1.size() - 1), s2),
			levenstein_recursion(s1, s2.substr(0, s2.size() - 1))) + 1;
			if (s1[s1.size() - 1] == s2[s2.size() - 1])
				return min(ans, levenstein_recursion(s1.substr(0, s1.size() - 1), s2.substr(0, s2.size() - 1)));
			else
				return min(ans, levenstein_recursion(s1.substr(0, s1.size() - 1), s2.substr(0, s2.size() - 1)) + 1);
		}
	}
\end{lstlisting}
\begin{lstlisting}[caption={Вызываемая функция рекурсивного поиска расстояния Левенштейна с мемоизацией}, label=lst:lev_rec_mem_1]
	int levenstein_recursion_memoization(const string &s1, const string &s2) {
		int n = static_cast<int>(s1.size()) + 1, m = static_cast<int>(s2.size()) + 1;
		if (s1.empty())
			return static_cast<int>(s2.size());
		else if (s2.empty())
			return static_cast<int>(s1.size());
		else {
			vector <vector<int>> dp(n, vector<int>(m, -1));
			return dp[s1.size()][s2.size()] = levenstein_recursion_memoization(s1, s2, dp);
		}
	}
\end{lstlisting}
\begin{lstlisting}[caption={Перегруженная функция рекурсивного поиска расстояния Левенштейна с мемоизацией}, label=lst:lev_rec_mem_2]
	int levenstein_recursion_memoization(const string &s1, const string &s2, vector <vector<int>> &dp) {
		if (dp[s1.size()][s2.size()] == -1) {
			if (s1.empty())
				return static_cast<int>(s2.size());
			else if (s2.empty())
				return static_cast<int>(s1.size());
			else {
				dp[s1.size()][s2.size()] = min(levenstein_recursion_memoization(s1.substr(0, s1.size() - 1), s2, dp),
				levenstein_recursion_memoization(s1, s2.substr(0, s2.size() - 1), dp)) +
				1;
				if (s1[s1.size() - 1] == s2[s2.size() - 1])
					return dp[s1.size()][s2.size()] = min(dp[s1.size()][s2.size()],
					levenstein_recursion(s1.substr(0, s1.size() - 1),
					s2.substr(0, s2.size() - 1)));
				else
					return dp[s1.size()][s2.size()] = min(dp[s1.size()][s2.size()],
					levenstein_recursion(s1.substr(0, s1.size() - 1),
					s2.substr(0, s2.size() - 1)) + 1);
			}
		}
		return dp[s1.size()][s2.size()];
	}
	
\end{lstlisting}
\clearpage
\begin{lstlisting}[caption={Функция нерекурсивного поиска расстояния Левенштейна}, label=lst:lev_iter]
	int levenstein(const string &s1, const string &s2) {
		int n = static_cast<int>(s1.size()) + 1,
		m = static_cast<int>(s2.size()) + 1;
		vector <vector<int>> dp(2, vector<int>(m));
		for (int j = 1; j < m; ++j)
		dp[0][j] = j;
		int cur = 1, prev = 0;
		for (int i = 1; i < n; ++i) {
			dp[cur][0] = i;
			for (int j = 1; j < m; ++j) {
				dp[cur][j] = min(dp[cur][j - 1], dp[prev][j]) + 1;
				if (s1[i - 1] == s2[j - 1])
					dp[cur][j] = min(dp[cur][j], dp[prev][j - 1]);
				else
					dp[cur][j] = min(dp[cur][j], dp[prev][j - 1] + 1);
			}
			cur ^= 1, prev ^= 1;
		}
		return dp[prev][m - 1];
	}
	
\end{lstlisting}
\begin{lstlisting}[caption={Функция нерекурсивного поиска расстояния Дамерау---Левенштейна}, label=lst:dam_lev_iter]
	int damerau_levenstein(const string &s1, const string &s2) {
		int n = static_cast<int>(s1.size()) + 1, m = static_cast<int>(s2.size()) + 1;
		vector <vector<int>> dp(3, vector<int>(m));
		for (int i = 1; i < m; ++i) {
			dp[0][i] = i;
		}
		int cur = 1, prev = 0, pprev = -1;
		for (int i = 1; i < n; ++i) {
			dp[cur][0] = i;
			for (int j = 1; j < m; ++j) {
				dp[cur][j] = min(dp[prev][j], dp[cur][j - 1]) + 1;
				if (s1[i - 1] == s2[j - 1])
					dp[cur][j] = min(dp[cur][j], dp[prev][j - 1]);
				else
					dp[cur][j] = min(dp[cur][j], dp[prev][j - 1] + 1);
				if (i > 1 && j > 1 && s1[i - 1] == s2[j - 2] && s1[i - 2] == s2[j - 1])
					dp[cur][j] = min(dp[cur][j], dp[pprev][j - 2] + 1);
			}
			cur = static_cast<int>((cur + 1) % dp.size()),
			prev = static_cast<int>((prev + 1) % dp.size()),
			pprev = static_cast<int>((pprev + 1) % dp.size());
		}
		return dp[prev][m - 1];
	}
	
\end{lstlisting}

\section{Тестирование}
Для тестирования алгоритмов, тестовые случаи были разбиты на классы эквивалентности. В таблице~\ref{tab:lev_tests} приведены тесты для алгоритма поиска расстояния Левенштейна, в таблице~\ref{tab:dam_lev_tests} --- для поиска расстояния Дамерау-Левенштейна.

\begin{table}[H]
	\centering
	\caption{Тестовые случаи для алгоритма поиска расстояния Левенштейна}
	\begin{tabular}{|c|c|c|c|}
		\hline
		\shortstack{Классы\\ эквивалентности} & Входные данные & \shortstack{Ожидаемые \\выходные данные} & \shortstack{Полученные\\ выходные данные} \\
		\hline
		Равные строки & \multicolumn{1}{l|}{\shortstack{<<abracadabra>>\\ <<abracadabra>>}} & 0 & 0\\
		\hline
		Пустая первая строка & \multicolumn{1}{l|}{\shortstack{<<>>\\ <<code>>}} & 4 & 4\\
		\hline
		Пустая вторая строка & \multicolumn{1}{l|}{\shortstack{<<code>>\\ <<>>}} & 4 & 4 \\
		\hline
		Замена букв & \multicolumn{1}{l|}{\shortstack{<<cap>>\\<<cat>>}} & 1 & 1\\
		\hline
		Добавление/удаление букв & \multicolumn{1}{l|}{\shortstack{<<fake>>\\ <<break>>}} & 4 & 4\\
		\hline 
	\end{tabular}
	\label{tab:lev_tests}
\end{table}
\begin{table}[H]
	\centering
	\caption{Тестовые случаи для алгоритма поиска расстояния Дамерау --- Левенштейна}
	\begin{tabular}{|c|c|c|c|}
		\hline
		\shortstack{Классы\\ эквивалентности} & Входные данные & \shortstack{Ожидаемые \\выходные данные} & \shortstack{Полученные\\ выходные данные} \\
		\hline
		Равные строки & \multicolumn{1}{l|}{\shortstack{<<abracadabra>>\\ <<abracadabra>>}} & 0 & 0\\
		\hline
		Пустая первая строка & \multicolumn{1}{l|}{\shortstack{<<>>\\ <<code>>}} & 4 & 4\\
		\hline
		Пустая вторая строка & \multicolumn{1}{l|}{\shortstack{<<code>>\\ <<>>}} & 4 & 4 \\
		\hline
		Замена букв & \multicolumn{1}{l|}{\shortstack{<<cap>>\\<<cat>>}} & 1 & 1\\
		\hline
		Добавление/удаление букв & \multicolumn{1}{l|}{\shortstack{<<fake>>\\ <<break>>}} & 4 & 4\\
		\hline 
		Транспозиция букв & \multicolumn{1}{l|}{\shortstack{<<cpu>>\\ <<cup>>}} & 1 & 1\\
		\hline
	\end{tabular}
	\label{tab:dam_lev_tests}
\end{table}
Для проведения тестирования использовалась библиотека Google Tests~\cite{gtest}. Пример модульного тестирования в листинге~\ref{lst:mod_tests}.
\clearpage
\begin{lstlisting}[caption={Тестовый набор и запуск модульных тестов}, label=lst:mod_tests]
	TEST(DAM_LEV, equal_string) {
		total_tests++;
		string s1 = "abracadabra", s2 = "abracadabra";
		int corr_ans = 0, ans = damerau_levenstein(s1, s2);
		ASSERT_EQ(corr_ans, ans);
	}
	
	TEST(DAM_LEV, empty_string_1) {
		total_tests++;
		string s1 = "", s2 = "code";
		int corr_ans = 4, ans = damerau_levenstein(s1, s2);
		ASSERT_EQ(corr_ans, ans);
	}
	
	TEST(DAM_LEV, empty_string_2) {
		total_tests++;
		string s1 = "code", s2 = "";
		int corr_ans = 4, ans = damerau_levenstein(s1, s2);
		ASSERT_EQ(corr_ans, ans);
	}
	
	TEST(DAM_LEV, swapped_letter_strings)
	{
		total_tests++;
		string s1 = "cap", s2 = "cat";
		int corr_ans = 1, ans = damerau_levenstein(s1, s2);
		ASSERT_EQ(corr_ans, ans);
	}
	
	TEST(DAM_LEV, complex_test_strings)
	{
		total_tests++;
		string s1 = "fake", s2 = "break";
		int corr_ans = 4, ans = damerau_levenstein(s1, s2);
		ASSERT_EQ(corr_ans, ans);
	}
	
	TEST(DAM_LEV, transposition_letters_strings)
	{
		total_tests++;
		string s1 = "cpu", s2 = "cup";
		int corr_ans = 1, ans = damerau_levenstein(s1, s2);
		ASSERT_EQ(corr_ans, ans);
	}
	
	int main(int argc, char **argv) {
		testing::InitGoogleTest(&argc, argv);
		int failed_cnt = RUN_ALL_TESTS();
		cout<<"TESTS_PASSED "<<total_tests-failed_cnt<<'\n'<<"TESTS_FAILED "<<failed_cnt<<'\n';
		return failed_cnt;
	}
\end{lstlisting}
\section*{Вывод}

Были реализованы алгоритмы поиска редакционных расстояний и проведено их тестирование. Все тесты были успешно пройдены.

\clearpage