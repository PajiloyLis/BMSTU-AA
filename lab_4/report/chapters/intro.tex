\ssr{ВВЕДЕНИЕ}

Зачастую в приложениях возникает необходимость выполнять некоторые действия параллельно. При этом исполняемый параллельно код должен иметь доступ общим для всей программы ресурсам. В связи с этим возникает необходимость в многопоточном коде. Кроме того, при выполнении на многопроцессорной машине такой код действительно будет выполняться параллельно, значительно ускоряя процесс вычислений~\cite{Tanenbaum}.

Цель данной лабораторной работы: получить навык организации параллельных вычислений на основе нативных потоков.

Задачи лабораторной работы:
\begin{enumerate}[label={\arabic*)}]
	\item разработать ПО осуществляющее выгрузку данных со страниц указанного ресурса;
	\item провести тестирование разработанного ПО;
	\item провести замеры времени выгрузки содержимого страниц при различном количестве потоков управления.
\end{enumerate}
\clearpage